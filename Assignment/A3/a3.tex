\RequirePackage[l2tabu,orthodox]{nag}  % warn about common LaTeX pitfalls
\RequirePackage[ascii]{inputenc}  % input is 7-bit ASCII
\RequirePackage{fixltx2e}  % fix LaTeX2e kernel bugs

\documentclass[11pt,twoside]{article}
\usepackage{color}
\usepackage{calc}  % arithmetic in length parameters
\usepackage{enumitem}  % more control over list formatting
\usepackage{fancyhdr}  % simpler headers and footers
\usepackage[margin=1in]{geometry}  % page layout
\usepackage{lastpage}  % for last page number
\usepackage{relsize}  % easier font size changes
\usepackage[normalem]{ulem}  % smarter underlining
\usepackage{url}  % verb-like typesetting of URLs
%\usepackage{xfrac}  % nicer looking simple fractions for text and math
\usepackage{amsmath}
\usepackage{amssymb}

% Set up fonts.
\usepackage[T1]{fontenc}  % use true 8-bit fonts
\usepackage{slantsc}  % allow slanted small-caps
\usepackage{microtype}  % perform various font optimizations
% Use Palatino-based monospace instead of kpfonts' default.
%\usepackage{newpxtext}
\ttfamily
\DeclareFontShape{T1}{\ttdefault}{m}{scsl}{<->ssub*\ttdefault/m/sc}{}
\DeclareFontShape{T1}{\ttdefault}{b}{scsl}{<->ssub*\ttdefault/b/sc}{}
% "Kepler" fonts.
\usepackage[nott,notextcomp]{kpfonts}
% Use curvier Latin Modern brackets instead of kpfonts' glyphs.
\DeclareSymbolFont{lmsymb}     {OMS}{lmsy}{m}{n}
\DeclareSymbolFont{lmlargesymb}{OMX}{lmex}{m}{n}
\DeclareMathDelimiter{\rbrace}{\mathclose}{lmsymb}{"67}{lmlargesymb}{"09}
\DeclareMathDelimiter{\lbrace}{\mathopen}{lmsymb}{"66}{lmlargesymb}{"08}

% Page layout: stretch text to fill up page.
\addtolength\footskip{.25\headheight}
\flushbottom

% Common list settings.

% Common macros.
%%  Common macros for the course CSC263H1 at the University of Toronto.
%%
%%  Copyright (c) 2014 Francois Pitt <fpitt@cs.utoronto.ca>
%%  last updated at 17:53 (EDT) on Sun 19 Oct 2014
%%
%%  CC BY-SA 4.0
%%  This work (the current file named 'macros-263.tex') is licensed under
%%  the Creative Commons Attribution-ShareAlike 4.0 International License.
%%  To view a copy of this license, visit
%%      http://creativecommons.org/licenses/by-sa/4.0/
%%  or send a letter to: Creative Commons, 444 Castro Street, Suite 900,
%%  Mountain View, California, 94041, USA.
%%  This is a human-readable summary of (and not a substitute for) the
%%  license.
%%  You are free to:
%%      Share -- copy and redistribute the material in any medium or format
%%      Adapt -- remix, transform, and build upon the material for any
%%          purpose, even commercially.
%%      The licensor cannot revoke these freedoms as long as you follow the
%%          license terms.
%%  Under the following terms:
%%      Attribution -- You must give appropriate credit, provide a link to
%%          the license, and indicate if changes were made. You may do so in
%%          any reasonable manner, but not in any way that suggests the
%%          licensor endorses you or your use.
%%      ShareAlike -- If you remix, transform, or build upon the material,
%%          you must distribute your contributions under the same license as
%%          the original.
%%      No additional restrictions -- You may not apply legal terms or
%%          technological measures that legally restrict others from doing
%%          anything the license permits.
%%  Notices:
%%      You do not have to comply with the license for elements of the
%%      material in the public domain or where your use is permitted by an
%%      applicable exception or limitation.
%%      No warranties are given. The license may not give you all of the
%%      permissions necessary for your intended use. For example, other
%%      rights such as publicity, privacy, or moral rights may limit how you
%%      use the material.

% Redefine \today in the style "DD Month YYYY".
\renewcommand*\today
 {\number\day\space\ifcase\month\or January\or February\or March\or
  April\or May\or June\or July\or August\or September\or October\or
  November\or December\fi\space\number\year}

% TeX trick so that math symbols are bold when text is.
\let\seiresfb\bfseries\def\bfseries{\boldmath\seiresfb}
\let\seiresdm\mdseries\def\mdseries{\unboldmath\seiresdm}

% Centered version of \llap and \rlap.
\providecommand*\clap[1]{\hbox to 0pt{\hss#1\hss}}

% Spacing macros with default argument.
\providecommand*\vfillstretch[1][1]{\vspace*{\stretch{#1}}}
\providecommand*\hfillstretch[1][1]{\hspace*{\stretch{#1}}}

% Abbreviations of latin phrases.
\let\latinabb\empty  % no special formatting
\providecommand*\ie{\latinabb{i.e.}}
\providecommand*\eg{\latinabb{e.g.}}
\providecommand*\etc{\latinabb{etc}}
\providecommand*\vs{\latinabb{vs}}

% General fonts and characters.
\let\longemph\textsl  % alternate way to emphasize
\let\strong\textbf  % strong emphasis
\let\code\texttt  % code fragments
\let\var\textsl  % multi-letter variables
\let\pred\mathbf  % predicates
\providecommand*\stup{\textsuperscript{st}}
\providecommand*\ndup{\textsuperscript{nd}}
\providecommand*\rdup{\textsuperscript{rd}}
\providecommand*\thup{\textsuperscript{th}}
\newcommand*\bbmath[1]{\ensuremath{\mathbb{#1}}}  % requires amssymb
\providecommand*\N{\bbmath{N}}  % requires amssymb
\providecommand*\Z{\bbmath{Z}}  % requires amssymb
\providecommand*\Q{\bbmath{Q}}  % requires amssymb
\providecommand*\R{\bbmath{R}}  % requires amssymb
\providecommand*\bigOh{\mathcal{O}}
\providecommand*\onehalf{\ensuremath{{}^1\!\!/\!_2}}
\providecommand*\letlinebreak{\penalty\exhyphenpenalty}
\providecommand*\halfthinspace{\kern.083333em}
\providecommand*\dash
   {\letlinebreak\halfthinspace---\halfthinspace\letlinebreak}
\let\per\slash  % for convenience

% For defined terms -- requires package ulem.
\newcommand*\defn[1]{\uline{\textit{#1}}}
\setlength{\ULdepth}{.2ex}

% General math macros.
\newcommand*\hiderel[1]{\mathrel{\hphantom{#1}}}
\providecommand*\comp[1]{\overline{#1}}  % set complement
\providecommand*\emptystr{\varepsilon}  % empty string
\providecommand*\lxor{\mathbin\oplus}  % exclusive or
\providecommand*\cat{\cdot}  % string concatenation
% Floors and ceilings, with optional size-changing command, e.g.,
% \floor[\Big]{x/2} becomes '\Big\lfloor{x/2}\Big\rfloor'.
\providecommand*\floor[2][]{{#1\lfloor}{#2}{#1\rfloor}}
\providecommand*\ceil[2][]{{#1\lceil}{#2}{#1\rceil}}
\providecommand*\lcm{\operatorname{lcm}}  % requires amsmath
\providecommand*\size{\operatorname{size}}  % requires amsmath
\providecommand*\len{\operatorname{len}}  % requires amsmath
\providecommand*\divides{\mathrel{|}}

% Redefined symbols from amssymb.
\renewcommand*\emptyset{\varnothing}  % rounder than default
\let\bigiff\iff
\renewcommand*\iff{\mathrel{\Leftrightarrow}}  % smaller than default
\let\bigimplies\implies
\renewcommand*\implies{\mathrel{\Rightarrow}}  % smaller than default
\renewcommand*\ge{\geqslant}  % with slanted line under the > sign
\renewcommand*\le{\leqslant}  % with slanted line under the < sign

% For algorithms, using either CLRS-style or Python-style pseudocode.
\let\ADT\textsc  % ADT names
\let\proc\textsc  % function names
\let\const\textsc  % constants (True, False, etc.)
\let\kw\textbf  % keywords (if, while, etc.)
\providecommand*\comm[1]{\textsl{\#\space#1}}  % comments
\providecommand*\opgets[1]{\mathrel{#1=}}
\providecommand*\eq{\mathrel{==}}
\providecommand*\True{\const{True}}
\providecommand*\False{\const{False}}
\providecommand*\None{\const{None}}
\providecommand*\cmod{\mathbin{\%}}
\providecommand*\nil{\const{nil}}

% Checkboxes and checklists.
% \checkmark already defined in amssymb
\makeatletter
\@ifpackageloaded{kpfonts}
   {\newcommand*\exmark{$\times$}}
   {\newcommand*\exmark{{\boldmath$\times$}}}
\makeatother
\newlength\exmarksize
\newlength\exmarkdepth
\newcommand*\checkbox[1][]
   {\settoheight\exmarksize{\exmark}
    \settodepth\exmarkdepth{\exmark}
    \addtolength\exmarksize{-\exmarkdepth}
    {\setlength\fboxrule{.1ex}\setlength\fboxsep{.1ex}%
    \fbox{\rule{0pt}{\exmarksize}\makebox[\exmarksize]{\smash{#1}}}}}
\newcommand*\checkedbox{\checkbox[\raisebox{.1ex}{\kern.2em$\checkmark$}]}
\newcommand*\exedbox{\checkbox[\exmark]}
\newenvironment*{checklist}{\begin{list}{\checkbox}{}}{\end{list}}
\newcommand*\checkeditem{\item[\checkedbox]}
\newcommand*\exeditem{\item[\exedbox]}
\newcommand*\boxeditem[1][]{\item[{\checkbox[#1]}]}

% Macros for drawing "underline" rules and half-boxes.
\providecommand*\urule[2][.5pt]{\rule[-.4ex]{#2}{#1}} % "underline" rule
% Underlined text box (with optional positioning argument).
\providecommand*\ubox[3][c]{\rlap{\urule{#2}}\makebox[#2][#1]{#3}}
% Underlined text box with caption (and optional caption positioning).
\providecommand*\capbox[5][c]{\ifx\empty#2\empty
    \else\rlap{\raisebox{-\baselineskip}{\makebox[#3][#1]{#2}}}\fi
    \ubox[#4]{#3}{#5}}

% For student numbers.
\providecommand*\hb{\urule{1.2em}}      % horizontal bar
\providecommand*\vb{\urule[1ex]{.5pt}}  % vertical bar
\providecommand*\studentnumberboxes     % now 10-digits long!
   {\vb\hb\vb\hb\vb\hb\vb\hb\vb\hb\vb\hb\vb\hb\vb\hb\vb\hb\vb\hb\vb}
\newlength\numberboxwidth
\settowidth\numberboxwidth{\studentnumberboxes}

% Command to format sample solutions for tutorials.
\newcommand\samplesolution[1]
   {\ifsolutions\begin{quote}\sffamily{#1}\end{quote}\fi}

% Marking-related macros.
\providecommand*\markitem[2][]
   {\item{\bfseries#2}\ifx\empty#1\empty\else
    \space[$#1$ mark\ifnum#1>1 s\fi]\fi:\quad\ignorespaces}
\providecommand*\errorcode[2][]
   {\item{\bfseries error code #2}\ifx\empty#1\empty\else
    \space[#1]\fi:\quad\ignorespaces}
\providecommand*\commonerror[1][]
   {\item{\bfseries common error}\ifx\empty#1\empty\else
    \space[#1]\fi:\quad\ignorespaces}
\makeatletter
\providecommand*\heading[1]%
   {\@startsection{heading}{9}{0pt}{-.75ex plus -1.5ex minus -.5ex}
    {-.5em plus -.5em minus -.25em}{\normalfont}*{\textsc{#1}}\mbox{}}
\makeatother
\newcommand*\st{\mathrel{|}}  % "such that" for set extension

% Headings.
\pagestyle{fancy}
\let\headrule\empty
\let\footrule\empty
\lhead{CSC\,263\,H1}
\chead{\large\scshape Assignment \#\,3}
\rhead{\scshape Fall 2015}
\lfoot{\scshape Dept.\@ of Computer Science, University of Toronto,
       St.~George Campus}
\cfoot{}
\rfoot{\scshape page \thepage\space of \pageref{LastPage}}


\begin{document}

\noindent
\strong{Worth:}  8\%
\hfill
\strong{Due:}  By 5:59pm on Tuesday 3 November\\[2ex]
\strong
   {Remember to write
    the \emph{full name} and \emph{student number}
    of \emph{every group member}
    prominently on your submission.}

\medskip

\noindent
\rule{\textwidth}{.5pt}\\[1ex]
\begingroup\slshape
    Please read and understand the policy on Collaboration
    given on the Course Information Sheet.
    Then, to protect yourself,
    list on the front of your submission
    \strong{every} source of information
    you used to complete this homework
    (other than your own lecture and tutorial notes).
    For example, indicate clearly
    the \strong{name} of every student from another group
    with whom you had discussions,
    the \strong{title and sections} of every textbook you consulted
    (including the course textbook),
    the \strong{source} of every web document you used
    (including documents from the course webpage),
    \etc.\par
        For each question, please write up detailed answers carefully.
    Make sure that you use notation and terminology correctly, and
    that you explain and justify what you are doing.
    Marks \strong{will} be deducted
    for incorrect or ambiguous use of notation and terminology, and
    for making incorrect, unjustified, ambiguous, or vague claims
    in your solutions.
\endgroup\\
\rule{\textwidth}{.5pt}
% If you use this file as a template for your solution, please remove
% (or comment out) the block above!

\begin{enumerate}[leftmargin=0pt]





\item 
Give an algorithm for the following problem. The input is $ $ a sequence
of $n$ numbers $\{ x_1, x_2, \ldots, x_n \}$, another sequence of
$n$ numbers $\{y_1, y_2, \ldots, y_n \}$, and a number $z$. Your
algorithm should determine whether or not
$z \in \{ x_i + y_j \mid 1 \leq i, j \leq n \}$. You should use
universal hashing families, and your algorithm should run in
expected time $O(n)$.

Provide justification that your algorithm is correct and runs
in the required time. Be very clear about which theorems from class
and/or the text you are using, and how.

\textcolor{red}{\sc Answer:}
\begin{itemize}[label = {}]
\item First use universal hash maps all $x_{i} \ (1\leq i \leq n)$ to $n$ brackets. 
 We know that {\sc Insert} for hashing takes $O(1)$; therefore, inserting $n$ keys would take $O(n)$.
\item Then we run search for $z-y_{j} \ (1\leq j \leq n)$. It is clearly {\sc Search} runs in expected time O(1) by the properties of universal hashing.
The worst case is $z \notin \{x_{i}+y_{j} |1\leq i,j \leq n\}$ which we have to loop through all possible  $y_{j} \ (1\leq j \leq n)$. 
Since each {\sc Search} run in expected time $O(1)$, then worst case run time is in $O(n)$.
\item  Algorithm should run in expected time $O(n)$.
\end{itemize}


\item
We can implement a queue $Q$ using two stacks $H$ and $T$ as follows:
Think of the stack $T$ as containing the ``tail'' of the queue
        (i.e., the recently inserted items),
        with the most recently inserted item at the top.
Think of the stack $H$ as containing the ``head'' of the queue
        (i.e., the older items),
        with the oldest item of the queue at the top.
Conceptually, $Q$ consists of $T$ and $H$ placed ``back-to-back''.
%       (see figure below).

To {\sc EnQueue} an item $x$ on $Q$, we actually {\sc Push} $x$ into $T$.
To {\sc DeQueue} an item, we {\sc Pop} $H$, provided $H$ is not empty.
If $H$ is empty, we transfer the items of $T$ to $H$ (by popping each item of $T$
        and then pushing it into $H$), and then {\sc Pop} $H$.


These algorithms are given below in pseudo-code.
(We assume that initially the stacks $H$ and $T$ are empty, and
        that the function {\sc StackEmpty}($T$) returns {\bf true} if $T$ is
        an empty stack and {\bf false} otherwise.)

\smallskip
\begin{tabbing}
MM\=MM\=MM\=MM\=MM\=MM\=MM\=MM\=MM\=MM\=MM\=MM\=\kill
{\sc EnQueue}$(Q,x)$ \>\>\>\>\>\>\>\> {\sc DeQueue}$(Q)$\\
\>{\sc Push}$(T,x)$  \>\>\>\>\>\>\>   \> {\bf if} {\sc StackEmpty}$(H)$ {\bf then}\\
                     \>\>\>\>\>\>\>\> \>\> {\bf loop}\\
                     \>\>\>\>\>\>\>\> \>\>\> {\bf exit when} {\sc StackEmpty}$(T)$\\
                     \>\>\>\>\>\>\>\> \>\>\> $x$ := {\sc Pop}$(T)$\\
                     \>\>\>\>\>\>\>\> \>\>\> {\sc Push}$(H,x)$\\
                     \>\>\>\>\>\>\>\> \>\> {\bf end loop}\\
                     \>\>\>\>\>\>\>\> \> {\bf end if}\\
                     \>\>\>\>\>\>\>\> \> {\bf return} {\sc Pop}$(H)$\\
\end{tabbing}
%}

\noindent
Assume that each {\sc Push}, {\sc Pop} and {\sc StackEmpty} operation takes
        $\Theta(1)$ time.

\begin{enumerate}
\item
What is the worst-case time complexity of a single operation
        in a sequence of $m$ {\sc EnQueue} and {\sc DeQueue} operations?
Derive matching upper and lower bounds.
That is, define an initial situation by describing what $H$ and $T$
look like at the start, and then define a sequence of $m$ operations,
where the sequence consists of {\sc EnQueue}'s and {\sc DeQueue}'s.
Then show that one of the operations in the sequence (probably the last operation) will have
the claimed worst-case time.
For the upper bound, show that no operation in any $m$-operation sequence can ever
take more time than the claimed worst-case time. \\

\textcolor{red}{\sc Answer:}
\begin{itemize}[label = {}]

\item Assume $T$ is a stack of $n$ numbers, and $H$ is a stack of $y$ numbers, such that $n \geq 0 \land y \geq 0$.
\item Assume there is a sequence of $m$ operations consisting of {\sc EnQueue}'s and {\sc DeQueue}'s.
\item Case 1, for {\sc EnQueue} operation:
	\begin{itemize}
	\item  Each  {\sc EnQueue} operation in $m$ takes $\Theta(1)$ because it is simply pushing a number onto stack $T$, which by assumption takes $\Theta(1)$.
	\end{itemize}
\item Case 2, for {\sc DeQueue} operation:
	\begin{itemize}
	\item case 2.1: If stack $H$ is not empty, then {\sc DeQueue} operation in $m$ will take $\Theta(1)$  since it is just  a {\sc Pop} operation which takes $\Theta(1)$.
	\item case 2.2: If stack $H$ is empty, then in the worst case, every operation except the last operation in $m$ is {\sc EnQueue}, then it will take $\Theta(2(m - 1 + n) + 1)=\Theta(m+n)$ time. The original $n$ number of elements in $T$ and the $(m-1)$ number of {\sc EnQueue}'s will be popped from $T$ and pushed into $H$, with each operation costing 2 executions. The last {\sc Pop} from $H$ will cost an additional 1 execution. \\
	\\
	This is the worst case because if {\sc DeQueue}  was not the last operation in $m$, then in case 2.2 {\sc DeQueue}  will take less than $2(m-1+n)+1$ operations . Let's assume that {\sc DeQueue} was $x$th operation in $m$, where $x$ is not the last operation. Then $x > 1$ and $2(m-1) > 2(m-x)$; hence,  $2(m-1+n) +1> 2(m-x+n)+1$.\\
	The worst-case  of a single operation happens when $H$ is empty and $T$ has $n$ elements and sequence of m operations are $[ {\sc EnQueue}, {\sc EnQueue}, {\sc EnQueue}.... {\sc EnQueue},  {\sc DeQueue}]$ ($m-1$ {\sc EnQueue}'s follow by 1 {\sc DeQueue}) and the time complexity is in $\Theta(m+n)$.
	\end{itemize}


\end{itemize}
\item
Use the accounting method to prove
that the amortised time complexity of each operation in 
a sequence of $m$ {\sc EnQueue} and {\sc DeQueue} operations is $O(1)$.

To solve this problem, first give a credit scheme indicating
how many credits to allocate to each EnQueue and DeQueue operation.
Secondly, state the credit invariant,
and thirdly, prove the credit invariant. \\
\textcolor{red}{\sc Answer:}
\begin{itemize}[label = {}]
\item Each  {\sc EnQueue} will charge 3 credits.
\item Each {\sc DeQueue}  will change 1 credits.
\item Each {\sc Push}  and {\sc Pop} cost 1 credit.
\item
\item Credit Invariant: At any step each element in $H$ will have 0 credit, each element in $T$ will have have 2 credits. \\ 
\item Proof by induction:
\item Base Case: Initially $H$ and $T$ are empty.
\item Induction Step: Assume all elements in $H$ have 0 credit, all elements $T$ have 2 credits.
	\begin{itemize}
		\item Case 1: {\sc EnQueue} \\
		When we call {\sc EnQueue}, $H$ remain unchanged. In $T$ {\sc EnQueue} charges 3 credits and cost 1 credit for {\sc Push} the element to $T$; hence the new item in $T$ has 2 credits.
		\item Case 2: {\sc DeQueue} \\
		 case 2.1: $H$ is not empty, then {\sc DeQueue} charges 1 credits and cost that 1 credit for {\sc Pop}  a element from $H$; hence,  the rest elements in $H$ remain unchanged. Also everything in $T$ remain unchanged.\\
		 case 2.2: $H$ is empty,  every item in $T$ are popped and pushed onto $H$ which cost all their 2 credits. Then using the 1 credit charges from {\sc DeQueue} to {\sc Pop} a element from $H$. Now $T$ is empty and every item in $H$ has 0 credit. 
	\end{itemize}
\end{itemize}
Thus invariant is always true, so total charge for sequence is upper bound on total on cost. Then total charge at most be $3n$ so amortized cost per operation is 3.

\end{enumerate}

\item


Recall that the doubling method enables the implementation of
a stack without placing a limit on the size of the stack, such that the
amortized complexity of each operation is $O(1)$. Every time the array
gets full, a new array is allocated whose size is twice the size of the
old array, and the old array is copied to the new array.

\begin{enumerate}
\item 

Suppose we change the implementation so that the size of the new array is 3/2 times the size of the old array. 
What is the time complexity of a sequence of m operations in the worst-case? Justify your answer.\\
\\
\textcolor{red}{\sc Answer:}

\begin{itemize}[label = {}]
\item Idea: Charge each operation 4 credits.
	\begin{itemize} 
	\item 1 credit  (the ``append-credit'') will be spent when appending the element.
	\item 1 credit (the ``copy-credit'') will be spent when copying the element to a new array.
	\item  2 credits (the ``recharge-credits'') are  used to recharge the old elements(first $\frac{2}{3}$ of the list) that have spent their ``copy-credits''.
	\end{itemize}	
\item Credit Invariant: Each element in last $\frac{1}{3}$  of the array has 3 credits.\\
	\begin{itemize}[label = {}]
	\item Proof by induction:
	\item Base Case: Initially array is empty.
	\item Inductive Step: 
		\begin{itemize}[label = {-}]
		\item  Array is not full, then 1 credits is spend for append and 3 credits stored on the item; hence, Invariant satisfied.
		\item  Array is full, then make a new array.  The last $\frac{1}{3}$  of the array using their ``copy-credit'' copy themselves to the new array 
		and  ``recharge-credits'' copy all items in first $\frac{2}{3}$ of array to the new list. Then add new item 3 credits. Since the  last $\frac{1}{3}$  of the array
		only have this newly added item with 3 credits; hence, Invariant satisfied.
		\end{itemize}	
	\end{itemize}
\item 
\item Therefore,  a charge of 4 credits is enough. Then time complexity of a sequence of $m$ operations in the worst-case is 4.
\end{itemize}

\item
Suppose we change the implementation so that the size of the new array is 50 plus the size of the old array. What is the time complexity of a sequence of m
operations in the worst-case? Justify your answer.\\
\\
\textcolor{red}{\sc Answer:}
\begin{itemize}[label = {}]
\item Idea: Charge each operation $\lceil \frac{n}{50} \rceil + 2$ credits, where n is the size of the old array.
	\begin{itemize} 
	\item 1 credit  (the ``append-credit'') will be spent when appending the element.
	\item 1 credit (the ``copy-credit'') will be spent when copying the element to a new array.
	\item  $\lceil \frac{n}{50} \rceil$ credits (the ``recharge-credits'') are  used to recharge the old elements (elements from old array) that have spent their ``copy-credits''.
	\end{itemize}
\item Credit Invariant: Each element in last $50$ elements of the array has $\lceil \frac{n}{50} \rceil + 1$ credits.\\
	\begin{itemize}[label = {}]
	\item Proof by induction:
	\item Base Case: Initially array is empty.
	\item Inductive Step: (suppose the size of the array is $n+50$, where $n$ is size of its previous array.)
		\begin{itemize}[label = {-}] 
		\item  Array is not full, then 1 credits is spend for append and $\lceil \frac{n}{50} \rceil + 1$ credits stored on the item; hence, Invariant satisfied.
		\item  Array is full, then make a new array.  The last $50$ elements of the array use their ``copy-credit'' copy themselves to the new array 
		and  ``recharge-credits'' copy all first $n$ items from array to the new list. Then add new item $\lceil \frac{n+50}{50} \rceil + 1$  credits. Since the  last 50 elements of the array
		only have this newly added item with $\lceil \frac{n+50}{50} \rceil + 1$ credits and $n+50$ is size of its previous array; hence, Invariant satisfied.
		\end{itemize}	
	\end{itemize}
\item 
\item Therefore,  a charge of $\lceil \frac{n}{50} \rceil + 2$ credits is enough. Then time complexity of a sequence of $m$ operations in the worst-case is $O(n)$.	
\end{itemize}

\end{enumerate}

\end{enumerate}

\end{document}